\usepackage{makeidx}
\makeindex
\usepackage[utf8]{inputenc}
\usepackage[T1]{fontenc}

\usepackage{amsmath}
\usepackage{amsfonts}
\usepackage{amsthm}
\usepackage{mathtools}
\usepackage{amssymb}
\usepackage[shortlabels]{enumitem}
\setlist[enumerate]{align=left}
\usepackage{indentfirst}
\usepackage{mathrsfs}
\usepackage[style=alphabetic-verb]{biblatex}
\addbibresource{references.bib}
\usepackage{tikz-cd,tikz}
    \usetikzlibrary{hobby, calc, intersections, decorations.markings, decorations.pathreplacing} %libraries
    \tikzset{>=latex} %sets -> to render as -latex arrow tip
                      %https://tex.stackexchange.com/questions/54796/how-to-set-default-style-for-arrow-tips-in-tikz
\input{insbox} %used for a floating diagram in sec 13 --deri https://tex.stackexchange.com/a/486033
\usepackage{quiver}
\usepackage{subfiles}

\usepackage[b5paper]{geometry} %page size, same as Blue Book


\newcommand{\plscite}[1]{{\color{purple} #1}} %Changed from green to purple to stop the eye massacre

\usepackage{chngcntr}

% allow section number to continue between chapters
\renewcommand{\thesection}{\arabic{section}}

% paragraph counter
\newcounter{parcounter}[section]
\setcounter{parcounter}{0}
\renewcommand{\theparcounter}{(\thesection.\Alph{parcounter})}
\newcommand{\newparagraph}{\refstepcounter{parcounter}\paragraph{\theparcounter}\label{\thesection.\Alph{parcounter}}}

\newcommand{\seclabel}[1]{}


\setlist[enumerate,1]{label=(\roman*)}




\newtheoremstyle{definitionpar}%             % Name
  {}%                                     % Space above
  {}%                                     % Space below
  {}%                                     % Body font
  {}%                                     % Indent amount
  {\bf}%                                  % Theorem head font
  {. }%                                     % Punctuation after theorem head
  { }%                                    % Space after theorem head, ' ', or \newline
  {{\bf \refstepcounter{parcounter}\theparcounter\label{\thesection.\Alph{parcounter}}\quad\thmname{#1}\thmnumber{  #2}\thmnote{ {\normalfont (#3)}}}}
\theoremstyle{definitionpar}
\newtheorem*{pardefinition}{Definition}
\newtheorem{partheorem}{Theorem}
\newtheorem{parproposition}{Proposition}[section]
\newtheorem{parlemma}{Lemma}[section]
\newtheorem{parcorollary}{Corollary}[section]
\newtheorem{parconvention}{Convention}
\newtheorem{parexample}{Example}[section]
\newtheorem{parexercise}{Exercise}[section]
\newtheorem{parexercises}{Exercises}[section]
\newtheorem{parremark}{Remark}[section]
\newtheorem{parapplication}{Application}[section]
\newtheorem*{pardefinition*}{Definition}
\newtheorem*{partheorem*}{Theorem}
\newtheorem*{parproposition*}{Proposition}
\newtheorem*{parlemma*}{Lemma}
\newtheorem*{parcorollary*}{Corollary}
\newtheorem*{parexamples}{Example}
\newtheorem*{parexercise*}{Exercise}
\newtheorem*{parexercises*}{Exercises}
\newtheorem*{parremark*}{Remark}



\newtheoremstyle{definition}%             % Name
  {}%                                     % Space above
  {}%                                     % Space below
  {}%                                     % Body font
  {}%                                     % Indent amount
  {\bf}%                                  % Theorem head font
  {. }%                                     % Punctuation after theorem head
  { }%                                    % Space after theorem head, ' ', or \newline
  {\thmname{#1}\thmnumber{  #2}\thmnote{ {\normalfont (#3)}}}
\theoremstyle{definition}
\newtheorem*{definition}{Definition}
\newtheorem{theorem}[partheorem]{Theorem}
\newtheorem{proposition}[parproposition]{Proposition}
\newtheorem{lemma}[parlemma]{Lemma}
\newtheorem{corollary}[parcorollary]{Corollary}
\newtheorem{example}[parexample]{Example}
\newtheorem{exercise}[parexercise]{Exercise}
\newtheorem{exercises}[parexercise]{Exercises}
\newtheorem{remark}[parremark]{Remark}
\newtheorem*{definition*}{Definition}      %numberless env's:
\newtheorem*{theorem*}{Theorem}
\newtheorem*{proposition*}{Proposition}
\newtheorem*{lemma*}{Lemma}
\newtheorem*{corollary*}{Corollary}
\newtheorem*{examples*}{Example}
\newtheorem*{exercise*}{Exercise}
\newtheorem*{exercises*}{Exercises}
\newtheorem*{remark*}{Remark}
\newtheorem*{conjecture*}{Conjecture}


% this is just for chapter 1 and 3 where there's a small paragraph after the chapter title
\renewenvironment{quote}{
  \noindent     % No indent of first character.
  \topsep=0pt   % No additional separator before.
  \parskip=0pt  % No paragraph separator.
  \interlinepenalty=10000  % Prevent a page break to occur in a quote block.
  \leftskip=0pt 
  \rightskip=0pt 
  \parfillskip=0pt
  \itshape  % Make quotations in italic.
  \list{}{
    \leftmargin 5em   % this is the adjusting screw
    \rightmargin 5em  % this is the adjusting screw
  }
  \item
  \relax
  }
  {\endlist}


\newenvironment{implyenumerate}
    {\begin{enumerate}[align=left,leftmargin=!,labelindent=5pt,itemindent=-15pt]
    }
    {
    \end{enumerate}
    }
  



%solution for custom numbering in theorems and lemmata:
\newtheorem{innercustomgeneric}{\customgenericname}
\providecommand{\customgenericname}{}
\newcommand{\newcustomtheorem}[2]{%
  \newenvironment{#1}[1]
  {%
   \renewcommand\customgenericname{#2}%
   \renewcommand\theinnercustomgeneric{##1}%
   \innercustomgeneric
  }
  {\endinnercustomgeneric}
}
\newcustomtheorem{customthm}{Theorem}
\newcustomtheorem{customlemma}{Lemma}
\newcustomtheorem{customcor}{Corollary}
\newcustomtheorem{customprop}{Proposition}
\newcustomtheorem{customexample}{Examples}
\newcustomtheorem{customremark}{Remark}

\usepackage{setspace}
\setstretch{1.5}



\usepackage{titlesec}
\titleformat{\chapter}[hang]
{\scshape\fillast\LARGE\bfseries}
{\thechapter .}
{0.5em}
{}

\titleformat*{\section}{\large\bfseries}

\usepackage{fancyhdr}
\fancyhf{}
\renewcommand{\chaptermark}[1]{\markboth{#1}{}}
\fancyhead[LE]{\scshape Chapter \thechapter: \leftmark }
\renewcommand{\sectionmark}[1]{ \markright{#1}{} }
\fancyhead[RO]{\scshape Section \thesection: \rightmark }
\renewcommand{\headrulewidth}{0.3pt}
\renewcommand{\footrulewidth}{\iffootnote{0pt}{0.3pt}}

\fancyfoot[C]{\thepage} %the ref is just for proofreading/editing purposes, will be removed later.

\titleformat{\part}[display]
{\centering\scshape\huge}
{Part \thepart}
{0.5em}
{}






\usepackage{adjustbox}

% https://tex.stackexchange.com/questions/253910/reference-to-enumerate-item-with-manually-set-label
% this is just for chapter 34 that dumb reference
\newcounter{dummy}
\makeatletter
\newcommand\myitem[1][]{\item[#1]\refstepcounter{dummy}\def\@currentlabel{#1}}
\makeatother



%%%hyperref and Friends go at the end
\usepackage[bookmarks=true, colorlinks=true, linkcolor=blue, citecolor= red,hypertexnames=false]{hyperref}
\usepackage[all]{hypcap}
\newcommand{\parref}[1]{\hyperref[#1]{#1}} %to ref paragraphs.

% this lets you use asterisks, daggers, etc. as footnotes https://tex.stackexchange.com/questions/631768/how-to-use-a-symbol-in-a-footnote
\usepackage[symbol]{footmisc}